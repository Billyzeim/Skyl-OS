\chapter{Introduction}

Welcome to my journey of writing an operating system from scratch... and hopefully not going insane in the process.
My hope is that this book will drag some of you along for this ride and help your learn from my mistakes. 

When I started studying about operating systems (no earlier than 4 months ago) I found very few sources, most of which
claim this dark path my helpless soul is about to stride along will spoil every bit of happiness a mortal can have and 
divert me of my final destination, eventually rendering me unable to finish this journey.
And me being the optimist I am, I denied this fate and walked down that path only to find out they were pretty... well, right.

However, the majority of obstacles I have encountered are immediately correlated with the lack of resources.
I and am not fit to tell if this lack is pursued for financial purposes or genuinely created by the difficulties of the field,
nevertheless, it halts possible advancement.

This book will contain what I have learned so far and explain my mental process. I have found reading along someones
learning experience can ease the struggle of continuously emerging challenges by making available the thought process followed
to overcome them.

\section{Why Write Your Own OS?}

Learing about Operating Systems is challenging but trying to implement one is a totally different beast. The struggle of
finding out even how to start studying was what would for most people be a critical burden. 
Therefore, diving into OS development definitely requires some "getting your hands dirty" to understand.

In the process there is definitely a hilarious amount of details your learning journey could diverge towards learning, but 
there is luckily as much to gain in knowledge. From how a bootloader works to how the hardware of your computer is wired
together to perform basic tasks like handling signals of peripheral devices or doing math, how much there is to learn is only 
limited by your determination. After finishing this project you will walk out a totally different person: a low- level developer
or even engineer, dare I say. 

\section{What You'll Learn}

\begin{itemize}
  \item How bootloaders work and how to build one
  \item What real mode and protected mode is
  \item Writing a memory manager
  \item Implementing syscalls
\end{itemize}

\section{Who This Book Is For}

By now you should know that the purpose of this book is to make diving into OS development a little more beginner friendly.
When writting this book I mainly give my own prespective, as an Electrical Engineering and Computer Science student, therefore
I mainly imagine myself referring to people of virtually my technical background and knowledge. 

However, I belive that hope that this project will also benefit hobbyists and professors looking to incorporate low-level
OS development into their courses, by helping them understand what challenges students might face during studying this
subject and providing a simple enough prototype which students can understand, replicate or iterate on.

\section{What I Already Knew}

As was mentioned above I, myself, have some technical background and it though I am not an expert it would surely help
if the readers of this book had a similar to mine lever of understanding of things or above.

More specifically I had experience on:
\begin{itemize}
    \item High-level C programming
    \item Minimal understanding of assembly
    \item Hardware components and their job
    \item Introductory digital design
  \end{itemize}

\section{How To Approach This Book}

While reading this book you will find that me and you have a slightly different way of understanding things,
simply because that is statistically true. When I started implementing things at first nothing worked properly. 
My studying was not structured because of the lack of resources and most of my progress occured through trial and- most
importantly- error. 

Having said that, I hope it is clear that we will struggle in slightly different things for slightly
different reasons and this should not discourage you at all. When (not if) times get tough feel free to contact me, another 
contributor or a professor of yours to clarify questions.