\chapter{Introduction}

Welcome to my journey of writing an operating system from scratch... and hopefully not going insane in the process.
My hope is that this book will bring some of you along for this ride and help you learn from — and hopefully avoid — my mistakes. 

When I started studying about operating systems (no earlier than 4 months ago) I found very few sources, most of which
claim OS development is a dark path mere mortals should avoid.
And me being the optimist I am, I denied this fate and walked down that path only to find out they were pretty... well, right.

However, the majority of obstacles I have encountered are immediately correlated with the lack of resources.
I am not sure whether this scarcity is due to financial incentives or the genuine difficulty of the field, 
but it undeniably slows down potential progress.

This book will contain what I have learned so far and explain my mental process. I have found following someone's
learning experience can ease the beginner's struggle. The reader can overcome common challenges by following the 
thought process someone else used to tackle them. However, all material should be accompanied by
hands-on experience for best understanding.

\section{Why Write Your Own OS?}

Studying Operating Systems is challenging but trying to implement one is a totally different beast. Even realizing how to
start studying could critically discourage many people. 
Therefore, diving into OS development definitely requires some "getting your hands dirty" to understand.

In the process there is definitely an overwhelming amount of low-level detail you could encounter, 
but there is luckily as much to gain in knowledge. From how a bootloader works to how the hardware of your computer is wired
together to perform basic tasks like handling signals of peripheral devices or doing math, how much there is to learn is only 
limited by your determination. After finishing this project you will walk out a totally different person: a low-level developer
or even engineer, dare I say. 

\section{What You'll Learn}

\begin{itemize}
  \item How bootloaders work and how to build one
  \item What real mode and protected mode is
  \item Writing a memory manager
  \item Implementing syscalls
\end{itemize}

\section{Who This Book Is For}

By now you should know that the purpose of this book is to make diving into OS development a little more beginner friendly.
When writing this book I mainly give my own perspective, as an Electrical Engineering and Computer Science student, therefore
I mainly imagine myself addressing people of virtually my technical background and knowledge. 

However, I believe that this project will also benefit hobbyists and professors looking to incorporate low-level
OS development into their courses, by helping them understand what challenges students might face while studying this
subject and providing a simple enough prototype which students can understand, replicate or iterate on.

\section{What You Should Already Know}

As was mentioned above I, myself, have some technical background. Although I am not an expert it would surely help
if the readers of this book had a level of understanding similar to or greater than mine.

More specifically I had experience in:
\begin{itemize}
    \item High-level C programming
    \item Minimal understanding of assembly
    \item Basic hardware knowledge and how components interact
    \item Introductory digital design
  \end{itemize}

Understanding the material of this book should require some familiarity with the above.

\section{How To Approach This Book}

While reading this book you will find that you and I have a slightly different way of understanding things,
simply because that is statistically true. When I started implementing things, nothing worked. 
My study process was chaotic due to the lack of structured resources. Most of my progress came from repeated 
trial — and more importantly — error.

Having said that, I hope it is clear that we will struggle in slightly different things for slightly
different reasons and this should not discourage you at all. When (not if) times get tough feel free 
to reach out to me, a contributor, or one of your professors to help clarify misunderstandings.

\section{Book Structure and Approach}

This book follows the narrative of my own development journey. In each stage of building the operating 
system, I had to make structural and architectural decisions. In the first chapters, I’ll walk you through 
the reasoning behind my choices — from setting up a simple bootloader and transitioning from 16-bit to 
32-bit protected mode, to handling interrupts and implementing memory management. These early chapters 
favor simplicity and clarity. Later on, we’ll explore more complex or alternative designs, compare 
architectures, and analyze the trade-offs behind them.